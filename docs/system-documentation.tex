\documentclass[12pt,a4paper]{article}
\usepackage[utf8]{inputenc}
\usepackage[margin=2.5cm]{geometry}
\usepackage{graphicx}
\usepackage{float}
\usepackage{hyperref}
\usepackage{listings}
\usepackage{xcolor}
\usepackage{titlesec}
\usepackage{fancyhdr}
\usepackage[bahasa]{babel}

% Custom colors
\definecolor{codegreen}{rgb}{0,0.6,0}
\definecolor{codegray}{rgb}{0.5,0.5,0.5}
\definecolor{codepurple}{rgb}{0.58,0,0.82}
\definecolor{backcolour}{rgb}{0.95,0.95,0.92}

% Code listing style
\lstdefinestyle{mystyle}{
    backgroundcolor=\color{backcolour},   
    commentstyle=\color{codegreen},
    keywordstyle=\color{blue},
    numberstyle=\tiny\color{codegray},
    stringstyle=\color{codepurple},
    basicstyle=\ttfamily\footnotesize,
    breakatwhitespace=false,         
    breaklines=true,                 
    captionpos=b,                    
    keepspaces=true,                 
    numbers=left,                    
    numbersep=5pt,                  
    showspaces=false,                
    showstringspaces=false,
    showtabs=false,                  
    tabsize=2
}
\lstset{style=mystyle}

% Header and footer
\pagestyle{fancy}
\fancyhf{}
\fancyhead[L]{Analisis Sistem \textit{Website} 8EH Radio ITB}
\fancyhead[R]{\thepage}
\renewcommand{\headrulewidth}{0.4pt}

% Title formatting
\titleformat{\section}{\Large\bfseries}{\thesection}{1em}{}
\titleformat{\subsection}{\large\bfseries}{\thesubsection}{1em}{}

\title{\textbf{Analisis Use Case Diagram, Class Diagram, dan Sequence Diagram dari Sistem Website 8EH Radio ITB}}
\author{Arqila Surya Putra - 18223047}
\date{\today}


\begin{document}

\maketitle
\newpage
\tableofcontents
\newpage

\section{Gambaran Umum Sistem}

8EH Radio ITB adalah\textit{ platform streaming} radio yang menyediakan layanan \textit{streaming} radio langsung, manajemen \textit{podcast}, konten blog, dan layanan \textit{shortener} URL. Sistem ini dibangun menggunakan Next.js dengan database MongoDB dan mengimplementasikan kontrol akses berbasis peran untuk fungsi-fungsi administratif.

\subsection{Fitur Utama}
\begin{itemize}
    \item \textit{Streaming} radio langsung dengan mekanisme \textit{fallback}
    \item Manajemen dan pemutaran \textit{podcast}
    \item Manajemen pembuatan dan \textit{publishing} blog
    \item \textit{Shortener} URL dengan \textit{analytics}
    \item Manajemen \textit{chart} musik (Tune Tracker)
    \item \textit{Dashboard} admin berbasis peran
    \item Autentikasi Google OAuth 2.0
    \item \textit{Proxy} audio untuk pengaksesan dan pemutaran \textit{file audio}.
\end{itemize}

\section{Arsitektur Sistem}

\subsection{Stack Teknologi}
\begin{itemize}
    \item \textbf{Frontend}: Next.js 14, React, Tailwind CSS
    \item \textbf{Backend}: Next.js API Routes, NextAuth.js
    \item \textbf{Database}: MongoDB dengan Prisma ORM
    \item \textbf{Autentikasi}: Google OAuth 2.0
    \item \textbf{Penyimpanan File}: Cloudflare R2
    \item \textbf{Pemrosesan Audio}: WaveSurfer.js, HTML5 Audio API
\end{itemize}

\subsection{Komponen Sistem}
Sistem terdiri dari beberapa komponen kunci:
\begin{itemize}
    \item \textbf{Komponen Frontend}: Komponen React untuk UI
    \item \textbf{Custom Hooks}: Logika yang dapat digunakan kembali untuk manajemen state
    \item \textbf{API Routes}: Endpoint backend untuk operasi data
    \item \textbf{Model Database}: Koleksi MongoDB melalui Prisma
    \item \textbf{Fungsi Utilitas}: Fungsi helper untuk peran dan database
\end{itemize}

\section{Diagram UML}

Sebelum membahas detail implementasi sistem, penting untuk memahami jenis-jenis diagram UML yang digunakan dalam dokumentasi ini. \textit{Unified Modeling Language} (UML) adalah bahasa pemodelan standar yang digunakan untuk menggambarkan, merancang, dan mendokumentasikan sistem perangkat lunak.

\subsection{Jenis Diagram yang Digunakan}

\subsubsection{Use Case Diagram}
Use Case Diagram menggambarkan interaksi antara pengguna (aktor) dengan sistem. Diagram ini menunjukkan:
\begin{itemize}
    \item \textbf{Aktor}: Pengguna atau sistem eksternal yang berinteraksi dengan sistem
    \item \textbf{Use Case}: Fungsi atau fitur yang dapat dilakukan oleh aktor
    \item \textbf{Relasi}: Hubungan antara aktor dan \textit{use case}
\end{itemize}

Dalam sistem 8EH Radio ITB, terdapat dua aktor utama:
\begin{itemize}
    \item \textbf{Guest}: Pengguna publik yang dapat menjelajahi konten dan mendengarkan radio/\textit{podcast}
    \item \textbf{Admin/Staff}: Pengguna terautentikasi dengan akses berbasis peran ke fungsi manajemen
\end{itemize}

\subsubsection{Class Diagram}
Class Diagram menggambarkan struktur statis sistem dengan menunjukkan:
\begin{itemize}
    \item \textbf{Class}: \textit{Template} untuk objek yang memiliki atribut dan metode
    \item \textbf{Atribut}: Properti atau data yang dimiliki oleh \textit{class}
    \item \textbf{Metode}: Operasi atau fungsi yang dapat dilakukan oleh \textit{class}
    \item \textbf{Relasi}: Hubungan antar \textit{class} (inheritance, association, composition, dll.)
\end{itemize}

Class Diagram sistem ini dibagi menjadi beberapa paket:
\begin{itemize}
    \item \textbf{Frontend Components}: React Component untuk \textit{interface} pengguna
    \item \textbf{Custom Hooks}: \textit{Hook} khusus untuk manajemen \textit{state}
    \item \textbf{API Routes}: \textit{Endpoint backend} untuk operasi data
    \item \textbf{Database Models}: Model data MongoDB
    \item \textbf{Utility Functions}: Fungsi bantu untuk sistem
\end{itemize}

\subsubsection{Sequence Diagram}
Sequence Diagram menggambarkan interaksi dinamis antara objek dalam sistem seiring waktu. Diagram ini menunjukkan:
\begin{itemize}
    \item \textbf{Participant}: Objek atau komponen yang berpartisipasi dalam interaksi
    \item \textbf{Lifeline}: Garis vertikal yang menunjukkan waktu hidup objek
    \item \textbf{Message}: Komunikasi antar objek
    \item \textbf{Activation}: Periode ketika objek aktif memproses pesan
\end{itemize}

Sequence diagram dalam dokumentasi ini mencakup 20 skenario utama yang mencakup:
\begin{itemize}
    \item Interaksi pengguna publik (Guest)
    \item Fungsi administratif (Admin/Staff)
    \item Alur autentikasi dan otorisasi
    \item Operasi CRUD untuk berbagai entitas
\end{itemize}

\section{Analisis Use Case}

Sistem mendukung 20 \textit{use case} inti yang mencakup akses publik dan fungsi administratif. \textit{Use case} ini dikelompokkan berdasarkan aktor yang menggunakannya dan kompleksitas fungsionalitasnya.

\begin{figure}[H]
    \centering
    \includegraphics[width=\textwidth]{images/use-case-diagram.png}
    \caption{Diagram Use Case Sistem 8EH Radio ITB}
    \label{fig:use-case}
\end{figure}

Diagram Use Case di atas menunjukkan semua interaksi yang dapat dilakukan oleh pengguna dengan sistem. Diagram ini menjadi fondasi untuk memahami kebutuhan fungsional sistem dan menjadi acuan dalam pengembangan fitur-fitur selanjutnya.

\section{Class Diagram}

Arsitektur sistem direpresentasikan melalui diagram kelas yang menunjukkan hubungan antara komponen \textit{frontend}, \textit{API routes}, model \textit{database}, dan fungsi utilitas.

\begin{figure}[H]
    \centering
    \includegraphics[width=\textwidth]{images/class-diagram.png}
    \caption{Diagram Class Sistem 8EH Radio ITB}
    \label{fig:class}
\end{figure}

Diagram Class ini memberikan gambaran detail tentang struktur internal sistem, termasuk:
\begin{itemize}
    \item \textbf{Komponen Frontend}: RadioPlayer, Navbar, Waveform, dan komponen UI lainnya
    \item \textbf{Custom Hooks}: useRadioStream, useGlobalAudio untuk manajemen state
    \item \textbf{API Routes}: Endpoint untuk operasi CRUD pada berbagai entitas
    \item \textbf{Model Database}: User, BlogPost, Podcast, ShortLink, dan model lainnya
    \item \textbf{Fungsi Utilitas}: roleUtils untuk kontrol akses dan prisma untuk \textit{database}
\end{itemize}

\section{Sequence Diagram}

Bagian ini menyajikan alur interaksi detail untuk \textit{use case} inti sistem. Diagram \textit{sequence} menunjukkan bagaimana komponen-komponen sistem berinteraksi dalam skenario tertentu, memberikan pemahaman yang mendalam tentang perilaku dinamis sistem.

\subsection{Pengelompokan Sequence Diagram}

\textit{Sequence diagram} dalam dokumentasi ini dikelompokkan menjadi dua kategori utama:

\subsubsection{Interaksi Pengguna Publik}
Diagram-diagram ini menunjukkan bagaimana pengguna Guest berinteraksi dengan sistem untuk mengakses fitur-fitur publik seperti mendengarkan radio, menjelajahi \textit{podcast}, dan menggunakan \textit{short link}.

\subsubsection{Fungsi Administratif}
Diagram-diagram ini menunjukkan alur kerja untuk fungsi-fungsi administratif yang memerlukan autentikasi dan otorisasi, seperti manajemen konten, konfigurasi sistem, dan manajemen pengguna.

\subsection{Interaksi Pengguna Publik}

\subsubsection{Pemutaran Radio Stream}
\begin{figure}[H]
    \centering
    \includegraphics[width=\textwidth]{images/sequence-1-play-radio-stream.png}
    \caption{Diagram Sequence Pemutaran Radio Stream}
    \label{fig:seq-play-radio}
\end{figure}

Diagram ini menunjukkan alur lengkap dari saat pengguna mengklik tombol play hingga audio radio mulai diputar. Proses melibatkan pengambilan konfigurasi \textit{stream}, pembuatan URL \textit{stream} yang unik, dan inisialisasi audio \textit{player}.

\subsubsection{Menjelajahi Podcast}
\begin{figure}[H]
    \centering
    \includegraphics[width=\textwidth]{images/sequence-4-browse-podcasts.png}
    \caption{Diagram Sequence Menjelajahi Podcast}
    \label{fig:seq-browse-podcasts}
\end{figure}

Alur ini menunjukkan bagaimana pengguna dapat melihat daftar \textit{podcast} yang tersedia dan memutar \textit{podcast} yang dipilih menggunakan sistem \textit{player} audio \textit{global}.

\subsubsection{Pemutaran Podcast}
\begin{figure}[H]
    \centering
    \includegraphics[width=\textwidth]{images/sequence-14-play-podcast.png}
    \caption{Diagram Sequence Pemutaran Podcast}
    \label{fig:seq-play-podcast}
\end{figure}

Diagram ini menunjukkan proses pemutaran \textit{podcast} yang melibatkan \textit{proxy} audio untuk keamanan file dan sinkronisasi dengan sistem audio global.

\subsubsection{Kontrol Pemutaran}
\begin{figure}[H]
    \centering
    \includegraphics[width=\textwidth]{images/sequence-8-control-playback.png}
    \caption{Diagram Sequence Kontrol Pemutaran}
    \label{fig:seq-control-playback}
\end{figure}

Alur kontrol pemutaran menunjukkan bagaimana pengguna dapat menghentikan audio dan mengatur volume melalui \textit{navigation bar} atau komponen \textit{player}.

\subsubsection{Menampilkan Tune Tracker}
\begin{figure}[H]
    \centering
    \includegraphics[width=\textwidth]{images/sequence-12-view-tune-tracker.png}
    \caption{Diagram Sequence Menampilkan Tune Tracker}
    \label{fig:seq-view-tune-tracker}
\end{figure}

Diagram ini menunjukkan bagaimana sistem menampilkan \textit{chart} musik dengan mengisi posisi kosong dengan \textit{placeholder} jika diperlukan.

\subsubsection{Membuka Short Link}
\begin{figure}[H]
    \centering
    \includegraphics[width=\textwidth]{images/sequence-6-open-shortlink.png}
    \caption{Diagram Sequence Membuka Short Link}
    \label{fig:seq-open-shortlink}
\end{figure}

Alur ini menunjukkan proses \textit{redirect short link} yang mencakup \textit{tracking} untuk \textit{analytics}.

\subsection{Fungsi Administratif}

\subsubsection{Sign in dengan Google}
\begin{figure}[H]
    \centering
    \includegraphics[width=\textwidth]{images/sequence-2-signin-google.png}
    \caption{Diagram Sequence Sign in dengan Google}
    \label{fig:seq-signin-google}
\end{figure}

Diagram ini menunjukkan proses autentikasi lengkap melalui Google OAuth 2.0, termasuk pengecekan \textit{whitelist email} dan pembuatan \textit{session}.

\subsubsection{Akses Dashboard}
\begin{figure}[H]
    \centering
    \includegraphics[width=\textwidth]{images/sequence-10-access-dashboard.png}
    \caption{Diagram Sequence Akses Dashboard}
    \label{fig:seq-access-dashboard}
\end{figure}

Alur ini menunjukkan proses otorisasi berbasis peran untuk mengakses \textit{dashboard} admin.

\subsubsection{Manajemen Podcast}
\begin{figure}[H]
    \centering
    \includegraphics[width=\textwidth]{images/sequence-3-manage-podcasts.png}
    \caption{Diagram Sequence Manajemen Podcast}
    \label{fig:seq-manage-podcasts}
\end{figure}

Diagram ini menunjukkan operasi CRUD untuk \textit{podcast}, termasuk pembuatan dan pengelolaan konten audio.

\subsubsection{Manajemen Blog Post}
\begin{figure}[H]
    \centering
    \includegraphics[width=\textwidth]{images/sequence-7-manage-blog-posts.png}
    \caption{Diagram Sequence Manajemen Blog Post}
    \label{fig:seq-manage-blog-posts}
\end{figure}

Alur ini menunjukkan manajemen \textit{posting} blog dengan relasi ke penulisnya yang bisa terdiri atas multi-\textit{writer} dan operasi CRUD lengkap.

\subsubsection{Manajemen Short Link}
\begin{figure}[H]
    \centering
    \includegraphics[width=\textwidth]{images/sequence-5-manage-shortlinks.png}
    \caption{Diagram Sequence Manajemen Short Link}
    \label{fig:seq-manage-shortlinks}
\end{figure}

Diagram ini menunjukkan pembuatan dan pengelolaan \textit{short link} dengan generasi \textit{slug} unik.

\subsubsection{Manajemen Tune Tracker}
\begin{figure}[H]
    \centering
    \includegraphics[width=\textwidth]{images/sequence-9-manage-tune-tracker.png}
    \caption{Diagram Sequence Manajemen Tune Tracker}
    \label{fig:seq-manage-tune-tracker}
\end{figure}

Alur ini menunjukkan pengelolaan \textit{chart} musik dengan operasi \textit{bulk update} untuk posisi \textit{chart}.

\subsubsection{Manajemen Konfigurasi Player}
\begin{figure}[H]
    \centering
    \includegraphics[width=\textwidth]{images/sequence-17-manage-player-config.png}
    \caption{Diagram Sequence Manajemen Konfigurasi Player}
    \label{fig:seq-manage-player-config}
\end{figure}

Diagram ini menunjukkan pengelolaan konfigurasi \textit{player} radio termasuk \textit{upload cover image}.

\subsubsection{Manajemen Konfigurasi Stream}
\begin{figure}[H]
    \centering
    \includegraphics[width=\textwidth]{images/sequence-18-manage-stream-config.png}
    \caption{Diagram Sequence Manajemen Konfigurasi Stream}
    \label{fig:seq-manage-stream-config}
\end{figure}

Alur ini menunjukkan pengelolaan konfigurasi \textit{streaming server} dan URL \textit{fallback}.

\subsubsection{Manajemen Pengguna \& Whitelist}
\begin{figure}[H]
    \centering
    \includegraphics[width=\textwidth]{images/sequence-19-manage-users-whitelist.png}
    \caption{Diagram Sequence Manajemen Pengguna \& Whitelist}
    \label{fig:seq-manage-users-whitelist}
\end{figure}

Diagram ini menunjukkan pengelolaan pengguna, peran, dan \textit{whitelist email} untuk kontrol akses.

\subsubsection{Menampilkan Analitik Dashboard}
\begin{figure}[H]
    \centering
    \includegraphics[width=\textwidth]{images/sequence-20-view-dashboard-analytics.png}
    \caption{Diagram Sequence Menampilkan Analitik Dashboard}
    \label{fig:seq-view-dashboard-analytics}
\end{figure}

Alur ini menunjukkan pengumpulan dan tampilan statistik sistem untuk \textit{dashboard} admin.

\section{Skema Database}

\subsection{Model Inti}
Sistem menggunakan MongoDB dengan model-model kunci berikut:

\subsubsection{Model User}
\begin{lstlisting}[language=javascript, caption=Skema Model User]
model User {
  id        String         @id @default(auto()) @map("_id") @db.ObjectId
  name      String?
  email     String?        @unique
  emailVerified DateTime?
  image     String?
  role      String         @default("KRU")
  createdAt DateTime       @default(now())
  authored  AuthorOnPost[]
  accounts  Account[]
  sessions  Session[]
  shortLinks ShortLink[]
  podcasts  Podcast[]
}
\end{lstlisting}

\subsubsection{Model BlogPost}
\begin{lstlisting}[language=javascript, caption=Skema Model BlogPost]
model BlogPost {
  id          String         @id @default(auto()) @map("_id") @db.ObjectId
  title       String
  slug        String         @unique
  content     String
  description String?
  readTime    String?
  category    String?
  mainImage   String?
  tags        String[]
  isFeatured  Boolean        @default(false)
  authors     AuthorOnPost[]
  createdAt   DateTime       @default(now())
  updatedAt   DateTime       @updatedAt
}
\end{lstlisting}

\subsubsection{Model Podcast}
\begin{lstlisting}[language=javascript, caption=Skema Model Podcast]
model Podcast {
  id          String   @id @default(auto()) @map("_id") @db.ObjectId
  title       String
  subtitle    String?
  description String
  date        String?
  duration    String?
  audioUrl    String
  image       String?
  coverImage  String?
  createdAt   DateTime @default(now())
  updatedAt   DateTime @updatedAt
  author      User     @relation(fields: [authorId], references: [id])
  authorId    String
}
\end{lstlisting}

\section{Endpoint API}

\subsection{Endpoint Publik}
\begin{itemize}
    \item \texttt{GET /api/blog} - Mengambil postingan blog
    \item \texttt{GET /api/podcast} - Mengambil \textit{podcast}
    \item \texttt{GET /api/tune-tracker} - Mengambil \textit{chart} musik
    \item \texttt{GET /api/stream-config} - Mendapatkan konfigurasi \textit{stream}
    \item \texttt{GET /api/redirect/[slug]} - \textit{Redirect short link}
    \item \texttt{GET /api/proxy-audio} - \textit{Proxy file audio}
\end{itemize}

\subsection{Endpoint Admin}
\begin{itemize}
    \item \texttt{POST /api/blog} - Membuat postingan blog
    \item \texttt{PUT /api/blog/[slug]} - Memperbarui postingan blog
    \item \texttt{POST /api/podcast} - Membuat \textit{podcast}
    \item \texttt{PATCH /api/podcast/[id]} - Memperbarui \textit{podcast}
    \item \texttt{POST /api/shortlinks} - Membuat \textit{short link}
    \item \texttt{PUT /api/shortlinks/[id]} - Memperbarui \textit{short link}
    \item \texttt{PATCH /api/tune-tracker} - Memperbarui \textit{tune tracker}
    \item \texttt{PATCH /api/users} - Memperbarui \textit{role} pengguna
\end{itemize}

\section{Fitur Keamanan}

\subsection{Autentikasi}
\begin{itemize}
    \item Integrasi Google OAuth 2.0
    \item \textit{Whitelist email} untuk kontrol akses
    \item Manajemen \textit{session} JWT
    \item Otorisasi berbasis peran
\end{itemize}

\subsection{Otorisasi}
Sistem mengimplementasikan kontrol akses berbasis peran dengan peran-peran berikut:
\begin{itemize}
    \item \textbf{DEVELOPER}: Akses penuh sistem
    \item \textbf{TECHNIC}: Akses konfigurasi teknis
    \item \textbf{REPORTER}: Manajemen \textit{blog} dan konten
    \item \textbf{MUSIC}: Manajemen musik dan \textit{podcast}
    \item \textbf{KRU}: Akses dasar
\end{itemize}

\section{Deployment \& Infrastruktur}

\subsection{Persyaratan Environment}
\begin{itemize}
    \item Node.js 18+
    \item \textit{Database} MongoDB
    \item Penyimpanan Cloudflare R2
    \item Kredensial Google OAuth
\end{itemize}

\subsection{Environment Variables}
\begin{lstlisting}[language=bash, caption=Environment Variables yang Diperlukan]
MONGODB_URL=mongodb://localhost:27017/8ehradio
NEXTAUTH_SECRET=your-secret-key
NEXTAUTH_URL=http://localhost:3000
GOOGLE_CLIENT_ID=your-google-client-id
GOOGLE_CLIENT_SECRET=your-google-client-secret
R2_ACCESS_KEY_ID=your-r2-access-key
R2_SECRET_ACCESS_KEY=your-r2-secret-key
R2_BUCKET_NAME=your-bucket-name
R2_ENDPOINT=your-r2-endpoint
\end{lstlisting}

\section{Kesimpulan}

Sistem 8EH Radio ITB menyediakan \textit{platform} yang lengkap untuk streaming radio, manajemen konten, dan fungsi-fungsi administratif. Arsitektur sistem mendukung skalabilitas, keamanan, dan kemudahan pemeliharaan melalui teknologi web modern dan praktik terbaik.

Dokumentasi yang disajikan dalam dokumen ini mencakup analisis sistem lengkap termasuk \textit{use case}, \textit{class diagram}, \textit{sequence diagram}, dan detail implementasi teknis yang diperlukan untuk pengembangan lanjutan yang akan dilakukan di tugas mata kuliah Teknologi Sistem Terintegrasi selanjutnya.

\end{document}